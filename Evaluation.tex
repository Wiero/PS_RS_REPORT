\chapter{Evaluation}
\label{cha:Evaluation}
Um die korrekte Funktionsweise und die Performance des ge"anderten Framestacks zu "uberpr"ufen wurde eine modifizierte AMIDAR Testbench verwendet. Vorallem im Fokus stand die korrekte Ausf"uhrung der "Spill and Fill" Abl"aufe und die Threadwechsel. 
\section{Testumgebung}
F"ur die Tests wurde die AMIDAR Testbench modifiziert um Funktionen mit einen großen Framestack oder Threadwechseln zu testen. Au{ss}erdem wurde die Hardware des Framestacks erweitert um die ben"otigte Anzahl an Takten f"ur einen bestimmten Vorgang aufzuzeichnen. 
\subsection{Hardware"anderungen zur Performance Evaluation}
Um die Performance evaluieren zu k"onnen ein zus"atzlicher Blockram angelegt in dem jeweils die Dauer eines bestimmten Vorgangs protokolliert wird. Es kann entweder die Dauer der Spill, der Fill oder der Threadswitch Vorg"ange protokolliert werden. 
Um auf diese Daten zuzugreifen wurden dem Framestack die Wishbone Register 15, 16, 17 und 18 hinzugef"ufgt. Mit den Register 18 wird dabei angegeben ob und welcher Vorgang protokolliert werden soll. Wird eine 0 in das Register geschrieben wird die Protokollfunktion deaktiviert. Mit Register 15 Wird angegeben aus welcher Adresse des Speichers gelesen werden soll. Die ausgelesenen Daten selber stehen in Register 18. Das Register 17 gibt die Anzahl der Eintr"age in dem Speicher an. 
Zur Protokollierung eines Vorgans werden die Takte gez"ahlt sobald der Startzustand eines Vorgangs erreicht ist und endet wenn der Vorgang beendet wurde und neue Tokens abgearbeitet werden k"onnen. Zu beachten dabei ist, das bei jeden Vorgang nur die Anzahl der Takte gemessen wird bei denen die Framestack FSM besch"aftig ist. Beim Spill zum Beispiel wird das z"ahlen beendet sobald die letzten Stackdaten in den Schreibfifo geschrieben wurde. Die Zeit, in der der AXI Bus belegt ist, ist etwas l"anger, da der FIFO noch abgearbeitet werden muss. Der Framestack selber kann währenddessen schon weitere Token abarbeiten.
Alle eben beschriebenen "Anderungen lassen sich wenn nicht benötigt durch eine Präprozessordefinition deaktivieren. 
\subsection{angepasste Testbench}
Um die weiterhin korrekte Funktionsweise der Grundlegenden Framestack Funktionalit"aten zu testen wurde die AMIDAR Testbench genutzt. Erweitert wurde diese dabei zum einen um Tests, die den Spill and Fill Vorgang testen. Daf"ur wurde eine Funktion ben"otigt, die einen m"oglichst gro{ss}en Stack erzeugt und dennoch eine einigerma{ss}en kurze Laufzeit hat. Die Funktion zur Rekursiven Berechnung der Fibonaci Folge erzeugt zwar schnell einen sehr großen Stack, allerdings ist die Laufzeit dieser deutlich zu lang um als Test in Frage zu kommen. Zum  Testen der Spill and Fill Funktionen wurde eine vereinfachte Ackermann Funktion genutzt, die mit 2 Parametern arbeitet.
%todo code
In dem neuen Testfall AckermannTest wird diese Funktion mit unterschiedlichen Parametern aufgerufen:
Ackermann (3,3):
				Rekursionstiefe 		= 63
				Maximale Stackgr"o{ss}e = 450 Wörter
				R"uckgabewert 			= 9	
				
Ackermann (3,4):
				Rekursionstiefe			= 127
				Maximale Stackgr"o{ss}e = 890 Wörter
				R"uckgabewert			= 61

Ackermann (3,5):
				Rekursionstiefe			= 127
				Maximale Stackgr"o{ss}e = 1785 Wörter
				R"uckgabewert			= 253

Weitere wichtige Testf"alle waren die schon vorhandenen, jedoch nicht einkommentierten Multithreading Tests. Genutzt wurde der "BasicThreadTest" und der "AdvancedThreadTest". Außerdem wurde eine Modifizierte Variante des BasicThreadTests genutzt, bei der der Ackermanntest in mehreren Threads ausgef"uhrt wurde. 

Eine weitere Anpassung der Testbench ist eine Methode die "uber Wishbone die im Framestackmodul gemessenen Performancedaten ausliest.  Zuerst wird die Anzahl der erfassten Vorg"ange ausgelesen. Anschließend wird "uber den Speicher mit den Messdaten iteriert und der Inhalt ausgegeben, dabei wird das arithmetische Mittel der Messwerte berechnet und ebenfalls ausgegeben. 


\section{Performance Messungen}
\subsection{Spill}
\subsection{Fill}
\subsection{Threadswitch}
