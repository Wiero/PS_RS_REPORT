\chapter{Formeln in Latex}
\label{cha:formeln}


Im Prinzip ist die Syntax relativ einfach aufgebaut. Wenn man eine Formel in LaTeX eingeben m�chte, muss eine entsprechende Umgebung gew�hlt werden. Die bekanntesten sind math und displaymath. 
\begin{itemize}
\item displaymath setzt die Formel ab und zentriert sie 
\item math baut die Formel mehr in den Text ein.
\end{itemize}

Eingeleitet wird eine solche Umgebung durch {\textbackslash begin} also z.B. {\textbackslash begin\{displaymath\}} und durch {\textbackslash end} beendet. Anhand folgenden Beispiels  l�sst sich zeigen wie ein Bruch aufbaut und dargestellt wird. \\
\\
Mit den Befehlen,\\
\\
\textbackslash begin\{math\}\\
\textbackslash frac\{1\}\{2\}  \\
\textbackslash end\{math\}\\
\\
wird der Bruch 
\begin{math} 
\frac{1}{2} 
\end{math} 
mitten im Text erscheinen. Gibt man die Befehle,\\
\\
\textbackslash begin\{displaymath\}\\
\textbackslash frac\{1\}\{2\}  \\
\textbackslash end\{displaymath\}\\
\\
erscheint der Bruch zentriert als separate Formel, n�mlich,
\begin{displaymath} 
\frac{1}{2} 
\end{displaymath} 
 

Andere Konstruktionselemente lassen sich besser an konkreten Formeln darstellen. Daher hier einige bekannte Formel aus Naturwissenschaft und Statistik.
	
\vspace{1cm} 
Formel f�r das arithmetische Mittel einer Stichprobe: 
\begin{displaymath} 
\bar{x} = \frac{1}{n} \sum_{i=1}^{n} x_i 
\end{displaymath} 
Formel f�r die Varianz einer Stichprobe: 
\begin{displaymath} 
s^2 = \frac{1}{n-1} \sum_{i=1}^{n} (x_i - \bar{x}) 
\end{displaymath} 
Formel f�r die Standardabweichung: 
\begin{displaymath} 
s = \sqrt{s^2} = \sqrt{\frac{1}{n-1} \sum_{i=1}^{n} (x_i - \bar{x})^2} 
\end{displaymath} 
oder: 
\begin{displaymath} 
= \sqrt{\frac{ \sum_{i=1}^{n} (x_i - \bar{x})^2}{n-1}} 
\end{displaymath} 
Geometrisches Mittel: 
\begin{displaymath} 
G = \sqrt[n]{ \prod^n_{i=1} x_i} 
\end{displaymath} 
Binomialkoeffizient: 
\begin{displaymath} 
{n \choose k} = \frac{ n! }{ k! (n-k) !} 
\end{displaymath} 
Zeitunabh�ngige dreidimensionale Schr�dingergleichung: 
\begin{displaymath} 
\frac{\partial^2 \psi}{\partial x ^2} + \frac{\partial^2 \psi}{\partial y ^2} + \frac{\partial^2 \psi}{\partial z ^2} = - \frac{2m}{\hbar^2}(E -U)\psi. 
\end{displaymath} 
Faradaysches Induktionsgesetz: 
\begin{displaymath} 
\oint \bf{E} \cdot ds = -\int \frac{\partial \mathcal{B}}{\partial t} \cdot \bf{A} 
\end{displaymath} 

